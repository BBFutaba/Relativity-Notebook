\documentclass[12pt]{article}
\usepackage{amsmath, amssymb, amsthm}
\usepackage{latexsym, epsfig, ulem, cancel, multicol, hyperref}
\usepackage{graphicx, tikz, subfigure,pgfplots}
\usepackage[a4paper, total={6in, 8in}]{geometry}
\setlength{\parindent}{0pt}
\usepackage{multirow}
\usepackage{mathtools}
\pgfplotsset{width=10cm,compat=1.9}
\usepackage{esint}

\title{Relativity Formula Sheet}
\author{Dennis Li}
\date{A formula sheet}


\newcommand{\liminfty}[1]{\lim_{#1 \to \infty}}
\newcommand{\limzero}[1]{\lim_{#1 \to 0}}
\newcommand{\Z}{\mathbb{Z}}
\newcommand{\R}{\mathbb{R}}
\newcommand{\C}{\mathbb{C}}
\newcommand{\lineint}[1]{\int_{#1}}
\newcommand{\pypx}[2]{\frac{\partial #1}{\partial #2}}
\newcommand{\divg}{\nabla \cdot}
\newcommand{\curl}{\nabla \times}
\newcommand{\dydx}[2]{\frac{d #1}{d #2}}
\newcommand{\sqbkt}[1]{\left[ #1 \right]}
\newcommand{\paren}[1]{\left( #1 \right)}
\newcommand{\tribkt}[1]{\left< #1 \right>}
\newcommand{\abso}[1]{\left|#1 \right|}


\begin{document}

\maketitle


\section*{Notation}
Speed of light: $c = 299792458 \; m/s \approx 3\times 10^{8}\; m/s$\\
\[
\Delta t = c\Delta t_{conv}
\]
Where $t_{conv}$ is the conventional time.
\[
\beta = \frac{v_{rel}}{c} \;\;\; \gamma = \frac{1}{\sqrt{1-\beta^2}}
\]
Where $v_{rel}$ is the relative speed between 2 frames, where observer in one of the frame is at rest.



\section{Special Relativity}
\subsection{Invariant spacetime interval}
\[
\Delta s^2 = -\Delta t^2 + \Delta r^2
\]
\subsection{Time Dilation}
\[
\Delta t = \gamma \Delta \tau
\]
$\Delta \tau $ is the proper time. This is the formula of time dilation. 

\subsection{Length Contraction}
\[
\Delta r = \frac{\Delta r_0}{\gamma}
\]
$\Delta r_0$ is the proper length.
\subsection{Doppler Effect of Light}
\[
f' = f \sqrt{\frac{1\pm\beta}{1\mp \beta}}
\]
if blue-shifted: $\frac{+}{-}$, and vice versa.
\subsection{Lorentz Transformation}
Explicit form of Lorentz transformation
\[
\begin{cases}
    t = \gamma(t'+\beta x')\\
    x = \gamma(\beta t' + x')\\
    y=y'\\
    z=z'
\end{cases}
\]
The Inverse Lorentz Transformation
\[
\begin{cases}
    t' = \gamma(t-\beta x)\\
    x = \gamma(-\beta t + x)\\
    y'=y\\
    z'=z
\end{cases}
\]
Matrix form
\[
\begin{bmatrix}
    t \\ x \\ y \\ z
\end{bmatrix}
=
\begin{bmatrix}
    \gamma      &   \beta\gamma   & 0 & 0\\
    \beta\gamma &   \gamma & 0 & 0\\
    0&0&1&0\\
    0&0&0&1
\end{bmatrix}
\begin{bmatrix}
    t'\\ x' \\ y' \\ z'
\end{bmatrix}
\]
And the inverse
\[
\begin{bmatrix}
    t' \\ x' \\ y' \\ z'
\end{bmatrix}
=
\begin{bmatrix}
    \gamma      &   -\beta\gamma   & 0 & 0\\
    -\beta\gamma &   \gamma & 0 & 0\\
    0&0&1&0\\
    0&0&0&1
\end{bmatrix}
\begin{bmatrix}
    t\\ x \\ y \\ z
\end{bmatrix}
\]
\subsection{Momentum-Energy}
\subsubsection{Momenergy derived}
\[
\textbf{Momenergy} = \paren{mass}\times \frac{\paren{\textbf{spacetime displacement}}}{\paren{\text{proper time of displacement}}}
\]
\subsubsection{Momentum and Energy}
\[
E=m\dydx{t}{\tau}
\]
\[
p_x = m\dydx{x}{\tau}\;\;\; p_y = m\dydx{y}{\tau}\;\;\; p_z = m\dydx{z}{\tau}
\]
\subsubsection{Total Energy}
\[
E_{tot} = K + m_{rest}
\]
E is the total energy and K is the kinetic energy of the particle , is the rest mass.
\[
E_{tot}^2 = E_{rest}^2 + p^2
\]




\end{document}

