\documentclass[12pt]{memoir}
\pagestyle{empty}
\usepackage{amsmath, amssymb, amsthm}
\usepackage{latexsym, epsfig, ulem, cancel, multicol, hyperref}
\usepackage{graphicx, tikz, subfigure,pgfplots}
\usepackage[margin=1in]{geometry}
\usepackage{multirow}
\usepackage{mathtools}
\usepackage{verbatim}
\usepackage{tikz}
\usepackage{pgfplots}


\setcounter{secnumdepth}{3}
\setlength{\parindent}{0pt}
\setlength{\parskip}{1ex}

\newcommand{\T}[0]{\top}
\newcommand{\F}[0]{\bot}
\newcommand{\liminfty}[1]{\lim_{#1 \to \infty}}
\newcommand{\limzero}[1]{\lim_{#1 \to 0}}
\newcommand{\limto}[1]{\lim_{#1}}
\newcommand{\Z}{\mathbb{Z}}
\newcommand{\R}{\mathbb{R}}
\newcommand{\C}{\mathbb{C}}
\newcommand{\Q}{\mathbb{Q}}
\newcommand{\odd}[0]{\mathbb{Z} - 2\mathbb{Z}}
\newcommand{\lineint}[1]{\int_{#1}}
\newcommand{\pypx}[2]{\frac{\partial #1}{\partial #2}}
\newcommand{\divg}{\nabla \cdot}
\newcommand{\curl}{\nabla \times}
\newcommand{\dydx}[2]{\frac{d #1}{d #2}}
\newcommand{\sqbkt}[1]{\left[ #1 \right]}
\newcommand{\paren}[1]{\left( #1 \right)}
\newcommand{\tribkt}[1]{\left< #1 \right>}
\newcommand{\abso}[1]{\left|#1 \right|}
\newcommand{\zero}{\{0\}}
\newcommand{\then}{\rightarrow}
\newcommand{\nonneg}{\Z^+ \cup \{0\}}
\DeclarePairedDelimiter\ceil{\lceil}{\rceil}
\DeclarePairedDelimiter\floor{\lfloor}{\rfloor}
\newcommand{\union}[2]{\bigcup_{#1}^{#2}}
\newcommand{\inter}[2]{\bigcap_{#1}^{#2}}
\newcommand{\openclose}[1]{\left( #1 \right]}
\newcommand{\closeopen}[1]{\left[ #1 \right)}
\newcommand{\compo}[2]{#1 e^{i #2}}
\newcommand{\laplase}{\bigtriangleup}
\newcommand{\bra}[1]{\left< #1 \right|}
\newcommand{\ket}[1]{\left| #1 \right>}
\newcommand{\braket}[2]{\left< #1 \mid #2 \right>}
\newcommand{\ketbra}[2]{\left| #1 \right> \left< #2 \right|}
\newcommand{\ketpsit}{\ket{\psi(t)}}
\newcommand{\ketphit}{\ket{\phi(t)}}
\newcommand{\ham}{\mathbf{H}}
\newcommand{\unx}{\hat{\mathbf{x}}}
\newcommand{\uny}{\hat{\mathbf{y}}}
\newcommand{\unz}{\hat{\mathbf{z}}}
\newcommand{\uni}{\hat{\mathbf{i}}}
\newcommand{\unj}{\hat{\mathbf{j}}}
\newcommand{\unk}{\hat{\mathbf{k}}}
\newcommand{\uns}{\hat{\mathbf{s}}}
\newcommand{\unr}{\hat{\mathbf{r}}}
\newcommand{\untheta}{\hat{\boldsymbol\theta}}
\newcommand{\unphi}{\hat{\boldsymbol\phi}}
\newcommand{\unrho}{\hat{\boldsymbol\rho}}




\newcommand{\wsnumber}{1}
\newcommand{\wstopic}{Vectors}
\pgfplotsset{
    every linear axis/.append style={
       axis x line=center,
       axis y line=center,
       xlabel={$x$},
       ylabel={$y$}
    },
    every axis plot/.append style={thick,mark=none}
}
\tikzset{
    point/.style={circle,draw,fill,minimum width=0.3ex,inner sep=0pt,outer sep=0pt},
    every label/.append style={black}
}


\usepackage[margin=1in]{geometry}
\usepackage{amsmath, amssymb, amsthm, graphicx, hyperref}
\usepackage{enumerate}
\usepackage{fancyhdr}
\usepackage{multirow, multicol}
\usepackage{tikz}
\pagestyle{fancy}
\fancyhead[RO]{Dennis Li}
\fancyhead[LO]{GR Independent Study}
\usepackage{comment}
\newif\ifshow
\showfalse

\ifshow
  \newenvironment{solution}{\textbf{Solution.}}{}
\else
  \excludecomment{solution}
\fi

\renewcommand{\thefootnote}{\fnsymbol{footnote}}
\usepackage{comment}


\newtheorem*{remark}{Remark}

\newcommand*{\GridSize}{4}

\newcommand*{\ColorCells}[1]{% #1 = list of x/y/color
  \foreach \x/\y/\color in {#1} {
    \node [fill=\color, draw=none, thick, minimum size=1cm] 
      at (\x-.5,\GridSize+0.5-\y) {};
    }%
}%

\begin{document}

\begin{center}
\ifshow
  \textbf{\Large GRW Workbook}\\
\else
  \textbf{\Large General Relativity Workbook}\\
\fi
Dennis Li\\Prof. Gabe\\
\end{center}

\hrule

\vspace{0.2cm}
\begingroup
\let\clearpage\relax
\chapter{Introduction}
\endgroup
There are no boxes in this chapter, but selected problem will be done.

\begingroup
\let\clearpage\relax
\chapter{Special Relativity}
\endgroup
\section{BOX 2.1}
\subsection{Exercise}
In the space below, prove the "only if" clause; that is, assume that S' is inertial and show that it must move at a constant velocity relative to S. Start by considering a free object at rest in S'.

\begin{proof}
    Suppose $S'$ is inertial and a \textbf{free} object is observed to at rest in $S'$. An object being free is mutually agreed upon by both $S$ and $S'$. By Newton's first law, the object will remain at rest in $S'$ since it is inertial by hypothesis. Now, if there is a second frame $S$ moving relative to $S'$, and observers in $S$ also observes the free object, then it will see the object moving along with $S'$ since the object is at rest with respect to $S'$. Since the object is free, it cannot participate in any interaction and therefore cannot change its state of motion. Therefore the only way for $S$ to be inertial is if it is moving at a constant velocity with respect to $S'$, otherwise it will observe a change in velocity in the free object.  
\end{proof}

\section{BOX 2.2}
\subsection{Exercise}
Verify that the following conversion is true. $M_\odot$ represents the solar mass.
\begin{align*}
g &= 1.09 \times 10^{-16} \, \text{m}^{-1} = 1 / (9.17 \times 10^{15} \, \text{m}) \approx 1 / (1 \, \text{ly})\\
G &= 7.426 \times 10^{-28} \, \text{m/kg} = 1477 \, \text{m} / M_\odot
\end{align*}
We start by verifying $g$. We start from $g = 9.8 \,\text{m/s$^2$}$. We introduce the conversion factor
\begin{align}
1\, \text{s} = 3\times 10^{8} \,\text{m}
\end{align}
The constant can be rewritten as
\begin{align}
9.8 \times \frac{\text{m}}{\paren{3\times 10^{8}}^2\text{m}^2} =  1.09 \times 10^{-16} \, \text{m}^{-1} = \frac{1}{1.09 \times 10^{-16}\,\text{m}}
\end{align}
We notice that
\begin{align}
1\,\text{ly} = \paren{365\times 24\times 60\times 60}\text{s} \times \paren{3\times 10^{8} \,\text{m/s}} \simeq 1\,\text{ly}
\end{align}
Therefore we arrived 
\begin{align}
g = 1 / (1 \, \text{ly})
\end{align}
Now we work on $G$. We know that 
\begin{align}
G = 6.67 \times 10^{-11} \frac{\text{N}\cdot\text{m}^2}{\text{kg}^2}
\end{align}
We know that \textit{N} can be expressed in SI unit as $1\,$N$= 1\frac{\text{kg}\cdot\text{m}}{\text{s}^2}$. And the units simplifies to
\begin{align}
G = 6.67 \times 10^{-11} \frac{\text{m}^3}{\text{kg}\cdot \text{s}^2}
\end{align}
We use the same conversion unit as before to rewrite the unit of time in meters, we will yield
\begin{align}
G = 6.67 \times 10^{-11} \frac{\text{m}}{\text{kg}\cdot \paren{3\times 10^{8}}^2}
\end{align}
This gives us
\begin{align}
G = 7.426 \times 10^{-28}\, \text{m/kg}
\end{align}
We know that the mass of the sun is
\begin{align}
M_\odot = 1.99 \times 10^{30}\,\text{kg}
\end{align}
Then we can rewrite the expression as
\begin{align}
G = 1477 \, \text{m} / M_\odot
\end{align}



\section{BOX 2.3}
    \subsection{Exercise}
    We are given the following two equations:

\begin{align}
t_E - x_E = \gamma (1 - \beta)(t'_E - x'_E) 
\end{align}
\begin{align}
t_E + x_E = \gamma (1 + \beta)(t'_E + x'_E) 
\end{align}

We are tasked with verifying the following equations:

\begin{align}
t_E = \gamma t'_E + \gamma \beta x'_E
\end{align}
\begin{align}
x_E = \gamma \beta t'_E + \gamma x'_E
\end{align}

Step 1:  Adding equations (2.13a) and (2.13b)

By adding equations (2.13a) and (2.13b), we get:
\begin{align}
(t_E - x_E) + (t_E + x_E) = \gamma (1 - \beta)(t'_E - x'_E) + \gamma (1 + \beta)(t'_E + x'_E)
\end{align}
Simplifying both sides:
\begin{align}
2t_E = \gamma \left( (1 - \beta)(t'_E - x'_E) + (1 + \beta)(t'_E + x'_E) \right)
\end{align}
Expanding:
\begin{align}
2t_E = \gamma \left( (1 - \beta)t'_E - (1 - \beta)x'_E + (1 + \beta)t'_E + (1 + \beta)x'_E \right)
\end{align}
\begin{align}
2t_E = \gamma \left( 2t'_E + 2\beta x'_E \right)
\end{align}
Dividing by 2:
\begin{align}
t_E = \gamma (t'_E + \beta x'_E)
\end{align}
Thus, we have verified the first equation:
\begin{align}
t_E = \gamma t'_E + \gamma \beta x'_E
\end{align}
Similarly, by subtracting equation (2.13b) from (2.13a)

Now subtracting equation (2.13b) from equation (2.13a), we get:
\begin{align}
(t_E - x_E) - (t_E + x_E) = \gamma (1 - \beta)(t'_E - x'_E) - \gamma (1 + \beta)(t'_E + x'_E)
\end{align}
Simplifying both sides:
\begin{align}
-2x_E = \gamma \left( (1 - \beta)(t'_E - x'_E) - (1 + \beta)(t'_E + x'_E) \right)
\end{align}
Expanding:
\begin{align}
-2x_E = \gamma \left( (1 - \beta)t'_E - (1 - \beta)x'_E - (1 + \beta)t'_E - (1 + \beta)x'_E \right)
\end{align}
\begin{align}
-2x_E = \gamma \left( -2\beta t'_E - 2x'_E \right)
\end{align}
Dividing by -2:
\begin{align}
x_E = \gamma \beta t'_E + \gamma x'_E
\end{align}
Thus, we have verified the second equation:
\begin{align}
x_E = \gamma \beta t'_E + \gamma x'_E
\end{align}


\subsection{Exercise}
We know that for a matrix $\Lambda$, and vectors $\mathbf{s}$ and $\mathbf{s}'$, the following is true
\begin{align}
\mathbf{s}' = \Lambda\mathbf{s} \then \Lambda^{-1}\mathbf{s}'=\mathbf{s}
\end{align}
Now we look for the inverse of the Lorenz transform matrix. Since the transformation does not concern $y$ and $z$ axis, we treat it as a 2 by 2 matrix.
\begin{align}
\Lambda = \begin{bmatrix}
    \gamma & -\gamma\beta\\
    -\gamma\beta & \gamma
\end{bmatrix}
\end{align}
The inverse of this matrix can be found as
\begin{align}
\Lambda^{-1} = \frac{1}{\abso{\Lambda}}\begin{bmatrix}
    \gamma & \gamma\beta\\
    \gamma\beta & \gamma
\end{bmatrix}
\end{align}
Where $\abso{\Lambda}$ is the determinent of $\Lambda$, and is defined as
\begin{align}
\abso{\Lambda} = \gamma^2 -\gamma^2\beta^2 = \gamma^2\paren{1-\beta^2}
\end{align}
But we know that $\frac{1}{\gamma^2} = 1 - \beta^2$, therefore
\begin{align}
\abso{\Lambda} = 1
\end{align}
And the inverse of $\Lambda$ is simply 
\begin{align}
\Lambda^{-1} =\begin{bmatrix}
    \gamma & \gamma\beta\\
    \gamma\beta & \gamma
\end{bmatrix}
\end{align}
Now we can add the unchanged $y$ and $z$ axis in to obtain the full inverse transformation.
\begin{align}
\mathbf{s} = \begin{bmatrix}
    \gamma & \gamma\beta & 0 & 0\\
    \gamma\beta & \gamma & 0 & 0\\
    0&0&1&0\\
    0&0&0&1
\end{bmatrix}\mathbf{s}'
\end{align}

\section{BOX 2.4}
No Exercise in this box. Acknowledge that Lorenz transform is the same as a hyperbolic rotation, defined by a rotation matrix
\begin{align}
\cosh\theta = \frac{1}{\sqrt{1-\beta^2}} \equiv \gamma \quad \sinh\theta = \frac{\beta}{\sqrt{1-\beta^2}}=\gamma \beta
\end{align}

\section{BOX 2.5}
\subsection{Exercise}
We would like to show that the \textit{spacetime interval} stays invariant throughout Lorenz transformation. First of all, we know that the spacetime interval is the magnitude of the 4-vector that we would denote as $\Delta s$. We know that the magnitude squared can be found by performing an inner product. We know that
\begin{align}
\mathbf{s'} = \Lambda\mathbf{s}
\end{align}
Therefore
\begin{align}
(\mathbf{s}')^2 = \paren{\mathbf{s}'}^T\mathbf{s} = \paren{\Lambda\mathbf{s}}^T\Lambda\mathbf{s} = \mathbf{s}^T\Lambda^T\Lambda\mathbf{s}
\end{align}
This would be true in 4-D Euclidean space, but we are working with a hyperbolic spacetime structure, so we have to incorporate another vector to adjust how we calculate the magnitude. The correct inner product should be
\begin{align}
\paren{\mathbf{s}'}^T\eta\mathbf{s}' = \paren{\Lambda\mathbf{s}}^T\eta \Lambda\mathbf{s} = \mathbf{s}^T\Lambda^T\eta \Lambda\mathbf{s}
\end{align}
Where 
\begin{align}
\eta = \begin{bmatrix}
    -1&0&0&0\\
    0&1&0&0\\
    0&0&1&0\\
    0&0&0&1
\end{bmatrix}
\end{align}
Now we can correctly obtain the magnitude of the 4-vector. Let's define
\begin{align}
\paren{\mathbf{s}'}^2 = \paren{\mathbf{s}'}^T\eta\mathbf{s}'
\end{align}
We can examine the term $\mathbf{s}^T\Lambda^T$. Since $\Lambda$ is symmetric, then we have 
\begin{align}
\mathbf{s}^T\Lambda^T = \begin{bmatrix}
    t&x&y&z
\end{bmatrix}\begin{bmatrix}
\gamma & -\gamma\beta & 0 & 0 \\
-\gamma\beta & \gamma & 0 & 0 \\
0 & 0 & 1 & 0 \\
0 & 0 & 0 & 1
\end{bmatrix} = \left[\begin{matrix}
t\gamma-x\beta\gamma & x\gamma-t\beta\gamma & y & z
\end{matrix}\right]
\end{align}
We see that this is simply
\begin{align}
\mathbf{s}^T\Lambda^T = \left[\begin{matrix}
t' & x' & y' & z'
\end{matrix}\right] = \paren{\mathbf{s}'}^T
\end{align}
Also, by definition, we have that
\begin{align}
\Lambda\mathbf{s} = s'
\end{align}
Therefore we have
\begin{align}
\paren{\mathbf{s}'}^T\eta\mathbf{s}' = \paren{\mathbf{s}'}^T\eta\mathbf{s}'
\end{align}
This shows that the magnitude of spacetime 4-vector stays the same before and after the Lorenz transformation. Consequently, $\Delta s$ stays the same. 

\section{BOX 2.6}
    \subsection{Exercise}
    We start by writing out the Lorenz transformation between frame $S$ and $S'$.
    \begin{align}
    \Delta x' = \gamma\paren{\Delta x - \beta \Delta t}
    \end{align}
    \begin{align}
    \Delta t' = \gamma\paren{\Delta t - \beta \Delta x}
    \end{align}
    If the order of two events happened in opposite order, we have $\Delta t' < 0$. This means that
    \begin{align}
    \gamma\paren{\Delta t - \beta \Delta x} < 0
    \end{align}
    Move things around, we have
    \begin{align}
    \frac{\Delta t}{\Delta x} < \beta 
    \end{align}
    This is certainly possible if $\Delta x$ is sufficiently large, or $\beta$ is sufficiently large. And we know this is allowed since 
    \begin{align}
    -\Delta t^2 + \Delta x^2 > 0
    \end{align}
    This tells us
    \begin{align}
    \Delta x^2 > \Delta t^2
    \end{align}
    But if we examine timelike interval, we have
    \begin{align}
    \Delta x^2 < \Delta t^2
    \end{align}
    This lead us back to where we were when we assumed $\Delta t' < 0$, or reversed order in time
    \begin{align}
    \frac{\Delta t}{\Delta x} > \beta 
    \end{align}
    We know this is not possible since $\beta \in \sqbkt{0,1}$, this simply means nothing can travel faster than light, since $\beta$ is defined to be $\frac{v}{c}$. And we have shown that spacelike events does not have causal relationship, therefore the sequence of which they happened can be reversed, and vice versa. 

    \section{BOX 2.7}
        \subsection{Exercise}
        Since we are interested in the proper time on the clock carried on this differential step, we have
        \begin{align}
        dt' = \sqrt{dt^2 - dx^2 -dy^2 -dz^2}
        \end{align}
        If we are to factor $dt^2$ out, we have
        \begin{align}
        dt' = dt\sqrt{1- \paren{\dydx{x^2}{t^2}+\dydx{y^2}{t^2}+\dydx{z^2}{t^2}}} = dt\sqrt{1-v^2}
        \end{align}

    \section{BOX 2.8}
        \subsection{Exercise}
        We can use Lorenz transformation to prove length contraction by defining 2 events happening at the same time in frame $S$. This means a separation of time $0$. We have Lorenz transformation
        \begin{align}
        \Delta x' = \gamma\paren{\Delta x - \beta\Delta t}
        \end{align}
        But we know that $\Delta t = 0$. Therefore
        \begin{align}
        \Delta x' = \gamma\Delta x
        \end{align}
        This means
        \begin{align}
        \Delta x = \frac{1}{\gamma}\Delta x' = \Delta x'\sqrt{1-\beta^2}
        \end{align}
        This shows that length is contracted compare to the proper length as measured in the frame where the events happened simultaneously. This means
        \begin{align}
        L = L_R\sqrt{1-\beta^2}
        \end{align}
\newpage
    \section{BOX 2.9}
        \subsection{Exercise}
        We can obtain $v_x$ by the following operation
        \begin{align}
        v'_x = \frac{dx'}{dt'}=\frac{\gamma(dx - \beta dt)}{\gamma(dt - \beta dx)} = \dydx{x}{t}\paren{\frac{1-\beta\frac{dt}{dx}}{1-\beta\frac{dx}{dt}}}= v_x\paren{\frac{1-\frac{\beta}{v_x}}{1-\beta v_x}} = \frac{v_x - \beta}{1 - \beta v_x}
        \end{align}
        We notice that this is exactly relativistic speed addition inverse, where we have previously defined. 
        \[
        v'_x = \frac{v_x - \beta}{1 - \beta v_x} \quad v_x = \frac{v'_x + \beta}{1 + \beta v'_x}
        \]

\begingroup
\let\clearpage\relax
\chapter{Four Vectors}
\endgroup
        \section{BOX 3.1}
            \subsection{Exercise}
            In this exercise, we want to show that
            \begin{align}
            \mathbf{a}\cdot \mathbf{b} = \mathbf{a}' \cdot \mathbf{b}'
            \end{align}
            To do so, use the Lorenz transformation to rewrite the 4 vector. 
            \begin{align}
            \mathbf{a}' = \mathbf{\Lambda}a \quad \mathbf{b}' = \mathbf{b}
            \end{align}
            Where $\Lambda$ is the Lorenz transform matrix. We have
            \begin{align}
            \mathbf{a}'\cdot \mathbf{b} = \Lambda\mathbf{a}\cdot \Lambda\mathbf{b}
            \end{align}
            This is essentially
            \begin{align}
            \paren{\Lambda \mathbf{a}}^T\eta\Lambda\mathbf{b}
            \end{align}
            Applying some properties of transpose, this is
            \begin{align}
            \paren{\mathbf{a}}^T\paren{\Lambda^T\eta\Lambda}\mathbf{b}
            \end{align}
            But we know that the Lorenz transform is preserved, and we have shown previously that $\Lambda^T\eta\Lambda = \eta$, we simply yield
            \begin{align}
            \mathbf{a}^T\eta\mathbf{b} = \mathbf{a}\cdot\mathbf{b}
            \end{align}
            We have shown that
            \begin{align}
            \mathbf{a}'\cdot\mathbf{b}' =\mathbf{a}\cdot\mathbf{b}
            \end{align}

        \section{BOX 3.2}
            \subsection{Exercise}
                Show that
                \begin{align}
                d\tau = \sqrt{-ds^2} = \sqrt{dt^2 - dx^2 - dy^2 - dz^2}
                \end{align}
                Use this to show that $\mathbf{u}\cdot\mathbf{u} = -1$. 
                We start by rewriting the proper time differential. And we have previously shown that
                \begin{align}
                d\tau = dt\sqrt{1-v^2}
                \end{align}
                Recall that the definition of $\mathbf{u}$, the 4 vector is that 
                \begin{align}
                \mathbf{u}= 
                \begin{bmatrix}
                u^t \\
                u^x \\
                u^y \\
                u^z
                \end{bmatrix}
                =
                \begin{bmatrix}
                \frac{dt}{dt\sqrt{1 - v^2}} \\
                \frac{dx}{dt\sqrt{1 - v^2}} \\
                \frac{dy}{dt\sqrt{1 - v^2}} \\
                \frac{dz}{dt\sqrt{1 - v^2}}
                \end{bmatrix}
                =
                \begin{bmatrix}
                \frac{1}{\sqrt{1 - v^2}} \\
                \frac{v_x}{\sqrt{1 - v^2}} \\
                \frac{v_y}{\sqrt{1 - v^2}} \\
                \frac{v_z}{\sqrt{1 - v^2}}
                \end{bmatrix}
                \end{align}
            Now we can try to define $\mathbf{u}\cdot\mathbf{u}$. We do this by
            \begin{align}
            \mathbf{u}\cdot\mathbf{u}= \mathbf{u}^T\eta \mathbf{u}
            \end{align}
            If we carry our this inner product, we end up with
            \begin{align}
            \mathbf{u}\cdot\mathbf{u} = \frac{-1 + v_x^2+v_y^2+v_z^2}{1-v^2} = -\frac{1-v^2}{1-v^2} = -1
            \end{align}
            We know that $v^2$ is the magnitude of velocity, which can be found by adding the square of each components. And this result is consistent with our expectation.

        \section{BOX 3.3}
        \subsection{Exercise}
        This is simply using binomial expansion to show that $\gamma$ can be approximated by $1+\frac{1}{v^2}$ by removing higher order terms and end up with $1+\frac{1}{2}v^2$, meaning the error is in order of $v^2$, where $v$ is actually $\beta$.

        \section{BOX 3.4}
        \subsection{Exercise}
        First we define frame $S$ to be the frame that sees 2 balls going at opposite direction. Then we define a frame $S'$ where mass 1 is stationary, or in other word, the frame is moving to the $+\unx$ direction at speed $v=\beta$. We can use the Speed transformation to find the speed of mass 2 in this new frame. 
        Since the motion is 1 dimensional and straight, $y$ and $z$ component of the velocity is not of our concern. The speed can be obtained by
        \begin{align}
        u'_2 = \frac{-\beta - \beta}{1-\paren{-\beta}\beta} = \frac{-2\beta}{1+\beta^2}
        \end{align}
        Use it again for mass 3, the collided mass, to obtain the final speed
        \begin{align}
        u'_3 = \frac{0-\beta}{1-0\cdot\beta} = -\beta
        \end{align}
        This finishes the proof. 
        
        \subsection{Exercise}
        Let us substitute the result into the given expression
        \begin{align}
        \frac{1}{\sqrt{1-\paren{u'_2}^2}} = \frac{1}{\sqrt{1-\paren{\frac{-2\beta}{1+\beta^2}}^2}}
        \end{align}
        We can break this down
        \begin{align}
            \frac{1}{\sqrt{1-\paren{\frac{-2\beta}{1+\beta^2}}^2}}\\
            =& \frac{1}{\sqrt{\paren{1+\frac{-2\beta}{1+\beta^2}}\paren{1-\frac{-2\beta}{1+\beta^2}}}}\\
            =&\frac{1}{\sqrt{\frac{(1-\beta)^2}{1+\beta^2}\cdot \frac{(1+\beta)^2}{1+\beta^2}}}\\
            =&\frac{1+\beta^2}{\paren{1-\beta}\paren{1+\beta}}\\
            =&\frac{1+\beta^2}{1-\beta^2}
        \end{align}
        We can now use this result to how the conservation of 4 momentum.
        \begin{align}
        \begin{bmatrix}
            m \\
            0 \\
            0 \\
            0
        \end{bmatrix}
        +
        \frac{1 + \beta^2}{1 - \beta^2} 
        \begin{bmatrix}
            m \\
            -m \left(\frac{-2\beta}{1 + \beta^2}\right) \\
            0 \\
            0
        \end{bmatrix}
        =
        \begin{bmatrix}
            m + \frac{(1 + \beta^2)m}{1 - \beta^2} \\
            \frac{(1 + \beta^2)(2m\beta)}{(1 + \beta^2)(1 - \beta^2)} \\
            0 \\
            0
        \end{bmatrix}.
        \end{align}
        
        Since \( u'_2 = \frac{-2\beta}{1 + \beta^2} \), the denominator cancels, and we are left with:
        
        \begin{align}
        \begin{bmatrix}
            m + \frac{(1 + \beta^2)m}{1 - \beta^2} \\
            \frac{2m\beta}{1 - \beta^2} \\
            0 \\
            0
        \end{bmatrix}.
        \end{align}
        
        Simplify the time component:
        
        \begin{align}
        \frac{m(1 - \beta^2) + m(1 + \beta^2)}{1 - \beta^2} = \frac{m(1 - \beta^2 + 1 + \beta^2)}{1 - \beta^2} = \frac{2m}{1 - \beta^2}.
        \end{align}
        
        Thus, the four-momentum equation simplifies to:
        
        \begin{align}
        \begin{bmatrix}
            \frac{2m}{1 - \beta^2} \\
            \frac{-2m\beta}{1 - \beta^2} \\
            0 \\
            0
        \end{bmatrix},
        \end{align}
        
        which matches exactly with the four-momentum of the combined mass \( M = 2m \) moving at \( u'_3 = -\beta \) in frame \( S' \):
        
        \begin{align}
        \frac{1}{\sqrt{1 - \beta^2}}
        \begin{bmatrix}
            M \\
            -Mv \\
            0 \\
            0
        \end{bmatrix}
        =
        \frac{1}{\sqrt{1 - \beta^2}}
        \begin{bmatrix}
            2m \\
            -2m\beta \\
            0 \\
            0
        \end{bmatrix}
        =
        \begin{bmatrix}
            \frac{2m}{1 - \beta^2} \\
            \frac{-2m\beta}{1 - \beta^2} \\
            0 \\
            0
        \end{bmatrix}.
        \end{align}

    \section{BOX 3.5}
    \subsection{Exercise}
    From the given expression, we can obtain the result
    \[
    2\mathbf{p}_p\cdot \mathbf{p}_\gamma = -2E_pE-2E_pE = -4E_pE
    \]
    We also know that 
    \[
    2\mathbf{p}_p\cdot \mathbf{p}_\gamma = -2m_pm_{\pi} -m_{\pi}^2
    \]
    We have obtained the equation
    \[
     -4E_pE = -2m_pm_{\pi} -m_{\pi}^2
    \]
    Therefore we can solve for $E_p$, this indeed gives us
    \[
    E_p \approx = \frac{m_{\pi}\paren{2m_p + m_{\pi}}}{4E}
    \]
    If we simply plug in the numbers, we have
    \[
    E_p\approx \frac{135\paren{2\cdot938+135}}{4\cdot 6.4\cdot 10^{-10}}\,\text{MeV} \approx 10^{14}\,\text{MeV} >> m_p
    \]

\begingroup
\let\clearpage\relax
\chapter{My Chapter}
\endgroup 
            
            








\end{document}