\documentclass[12pt]{article}
\pagestyle{empty}
\usepackage{amsmath, amssymb, amsthm}
\usepackage{latexsym, epsfig, ulem, cancel, multicol, hyperref}
\usepackage{graphicx, tikz, subfigure,pgfplots}
\usepackage[margin=1in]{geometry}
\setlength{\parindent}{0pt}
\usepackage{multirow}
\usepackage{mathtools}
\usepackage{verbatim}
\usepackage{tikz}
\usepackage{pgfplots}

\setlength{\parskip}{1ex}

\newcommand{\T}[0]{\top}
\newcommand{\F}[0]{\bot}
\newcommand{\liminfty}[1]{\lim_{#1 \to \infty}}
\newcommand{\limzero}[1]{\lim_{#1 \to 0}}
\newcommand{\limto}[1]{\lim_{#1}}
\newcommand{\Z}{\mathbb{Z}}
\newcommand{\R}{\mathbb{R}}
\newcommand{\C}{\mathbb{C}}
\newcommand{\Q}{\mathbb{Q}}
\newcommand{\odd}[0]{\mathbb{Z} - 2\mathbb{Z}}
\newcommand{\lineint}[1]{\int_{#1}}
\newcommand{\pypx}[2]{\frac{\partial #1}{\partial #2}}
\newcommand{\divg}{\nabla \cdot}
\newcommand{\curl}{\nabla \times}
\newcommand{\dydx}[2]{\frac{d #1}{d #2}}
\newcommand{\sqbkt}[1]{\left[ #1 \right]}
\newcommand{\paren}[1]{\left( #1 \right)}
\newcommand{\tribkt}[1]{\left< #1 \right>}
\newcommand{\abso}[1]{\left|#1 \right|}
\newcommand{\zero}{\{0\}}
\newcommand{\then}{\rightarrow}
\newcommand{\nonneg}{\Z^+ \cup \{0\}}
\DeclarePairedDelimiter\ceil{\lceil}{\rceil}
\DeclarePairedDelimiter\floor{\lfloor}{\rfloor}
\newcommand{\union}[2]{\bigcup_{#1}^{#2}}
\newcommand{\inter}[2]{\bigcap_{#1}^{#2}}
\newcommand{\openclose}[1]{\left( #1 \right]}
\newcommand{\closeopen}[1]{\left[ #1 \right)}
\newcommand{\compo}[2]{#1 e^{i #2}}
\newcommand{\laplase}{\bigtriangleup}
\newcommand{\bra}[1]{\left< #1 \right|}
\newcommand{\ket}[1]{\left| #1 \right>}
\newcommand{\braket}[2]{\left< #1 \mid #2 \right>}
\newcommand{\ketbra}[2]{\left| #1 \right> \left< #2 \right|}
\newcommand{\ketpsit}{\ket{\psi(t)}}
\newcommand{\ketphit}{\ket{\phi(t)}}
\newcommand{\ham}{\mathbf{H}}
\newcommand{\unx}{\hat{\mathbf{x}}}
\newcommand{\uny}{\hat{\mathbf{y}}}
\newcommand{\unz}{\hat{\mathbf{z}}}
\newcommand{\uni}{\hat{\mathbf{i}}}
\newcommand{\unj}{\hat{\mathbf{j}}}
\newcommand{\unk}{\hat{\mathbf{k}}}
\newcommand{\uns}{\hat{\mathbf{s}}}
\newcommand{\unr}{\hat{\mathbf{r}}}
\newcommand{\untheta}{\hat{\boldsymbol\theta}}
\newcommand{\unphi}{\hat{\boldsymbol\phi}}
\newcommand{\unrho}{\hat{\boldsymbol\rho}}



\newcommand{\wsnumber}{1}
\newcommand{\wstopic}{Vectors}
\pgfplotsset{
    every linear axis/.append style={
       axis x line=center,
       axis y line=center,
       xlabel={$x$},
       ylabel={$y$}
    },
    every axis plot/.append style={thick,mark=none}
}
\tikzset{
    point/.style={circle,draw,fill,minimum width=0.3ex,inner sep=0pt,outer sep=0pt},
    every label/.append style={black}
}


\usepackage[margin=1in]{geometry}
\usepackage{amsmath, amssymb, amsthm, graphicx, hyperref}
\usepackage{enumerate}
\usepackage{fancyhdr}
\usepackage{multirow, multicol}
\usepackage{tikz}
\pagestyle{fancy}
\fancyhead[RO]{Dennis Li}
\fancyhead[LO]{GR Independent Study}
\usepackage{comment}
\newif\ifshow
\showfalse

\ifshow
  \newenvironment{solution}{\textbf{Solution.}}{}
\else
  \excludecomment{solution}
\fi

\renewcommand{\thefootnote}{\fnsymbol{footnote}}
\usepackage{comment}


\newtheorem*{remark}{Remark}

\newcommand*{\GridSize}{4}

\newcommand*{\ColorCells}[1]{% #1 = list of x/y/color
  \foreach \x/\y/\color in {#1} {
    \node [fill=\color, draw=none, thick, minimum size=1cm] 
      at (\x-.5,\GridSize+0.5-\y) {};
    }%
}%

\begin{document}

\begin{center}
\ifshow
  \textbf{\Large GRW Workbook}\\
\else
  \textbf{\Large General Relativity Workbook}\\
\fi
Dennis Li\\Prof. Gabe\\
\end{center}

\hrule

\vspace{0.2cm}

\section{Introduction}
There are no boxes in this chapter, but selected problem will be done.

\section{Special Relativity}
\subsection{BOX 2.1}
\subsubsection{Exercise}
In the space below, prove the "only if" clause; that is, assume that S' is inertial and show that it must move at a constant velocity relative to S. Start by considering a free object at rest in S'.

\begin{proof}
    Suppose $S'$ is inertial and a \textbf{free} object is observed to at rest in $S'$. An object being free is mutually agreed upon by both $S$ and $S'$. By Newton's first law, the object will remain at rest in $S'$ since it is inertial by hypothesis. Now, if there is a second frame $S$ moving relative to $S'$, and observers in $S$ also observes the free object, then it will see the object moving along with $S'$ since the object is at rest with respect to $S'$. Since the object is free, it cannot participate in any interaction and therefore cannot change its state of motion. Therefore the only way for $S$ to be inertial is if it is moving at a constant velocity with respect to $S'$, otherwise it will observe a change in velocity in the free object.  
\end{proof}

\subsection{BOX 2.2}
\subsubsection{Exercise}
Verify that the following conversion is true. $M_\odot$ represents the solar mass.
\begin{align*}
g &= 1.09 \times 10^{-16} \, \text{m}^{-1} = 1 / (9.17 \times 10^{15} \, \text{m}) \approx 1 / (1 \, \text{ly})\\
G &= 7.426 \times 10^{-28} \, \text{m/kg} = 1477 \, \text{m} / M_\odot
\end{align*}
We start by verifying $g$. We start from $g = 9.8 \,\text{m/s$^2$}$. We introduce the conversion factor
\[
1\, \text{s} = 3\times 10^{8} \,\text{m}
\]
The constant can be rewritten as
\[
9.8 \times \frac{\text{m}}{\paren{3\times 10^{8}}^2\text{m}^2} =  1.09 \times 10^{-16} \, \text{m}^{-1} = \frac{1}{1.09 \times 10^{-16}\,\text{m}}
\]
We notice that
\[
1\,\text{ly} = \paren{365\times 24\times 60\times 60}\text{s} \times \paren{3\times 10^{8} \,\text{m/s}} \simeq 1\,\text{ly}
\]
Therefore we arrived 
\[
g = 1 / (1 \, \text{ly})
\]
Now we work on $G$. We know that 
\[
G = 6.67 \times 10^{-11} \frac{\text{N}\cdot\text{m}^2}{\text{kg}^2}
\]
We know that \textit{N} can be expressed in SI unit as $1\,$N$= 1\frac{\text{kg}\cdot\text{m}}{\text{s}^2}$. And the units simplifies to
\[
G = 6.67 \times 10^{-11} \frac{\text{m}^3}{\text{kg}\cdot \text{s}^2}
\]
We use the same conversion unit as before to rewrite the unit of time in meters, we will yield
\[
G = 6.67 \times 10^{-11} \frac{\text{m}}{\text{kg}\cdot \paren{3\times 10^{8}}^2}
\]
This gives us
\[
G = 7.426 \times 10^{-28}\, \text{m/kg}
\]
We know that the mass of the sun is
\[
M_\odot = 1.99 \times 10^{30}\,\text{kg}
\]
Then we can rewrite the expression as
\[
G = 1477 \, \text{m} / M_\odot
\]



\subsection{BOX 2.3}
    \subsubsection{Exercise}
    We are given the following two equations:

\[
t_E - x_E = \gamma (1 - \beta)(t'_E - x'_E) \tag{2.13a}
\]
\[
t_E + x_E = \gamma (1 + \beta)(t'_E + x'_E) \tag{2.13b}
\]

We are tasked with verifying the following equations:

\[
t_E = \gamma t'_E + \gamma \beta x'_E
\]
\[
x_E = \gamma \beta t'_E + \gamma x'_E
\]

### Step 1: Adding equations (2.13a) and (2.13b)

By adding equations (2.13a) and (2.13b), we get:
\[
(t_E - x_E) + (t_E + x_E) = \gamma (1 - \beta)(t'_E - x'_E) + \gamma (1 + \beta)(t'_E + x'_E)
\]
Simplifying both sides:
\[
2t_E = \gamma \left( (1 - \beta)(t'_E - x'_E) + (1 + \beta)(t'_E + x'_E) \right)
\]
Expanding:
\[
2t_E = \gamma \left( (1 - \beta)t'_E - (1 - \beta)x'_E + (1 + \beta)t'_E + (1 + \beta)x'_E \right)
\]
\[
2t_E = \gamma \left( 2t'_E + 2\beta x'_E \right)
\]
Dividing by 2:
\[
t_E = \gamma (t'_E + \beta x'_E)
\]
Thus, we have verified the first equation:
\[
t_E = \gamma t'_E + \gamma \beta x'_E
\]
Similarly, by subtracting equation (2.13b) from (2.13a)

Now subtracting equation (2.13b) from equation (2.13a), we get:
\[
(t_E - x_E) - (t_E + x_E) = \gamma (1 - \beta)(t'_E - x'_E) - \gamma (1 + \beta)(t'_E + x'_E)
\]
Simplifying both sides:
\[
-2x_E = \gamma \left( (1 - \beta)(t'_E - x'_E) - (1 + \beta)(t'_E + x'_E) \right)
\]
Expanding:
\[
-2x_E = \gamma \left( (1 - \beta)t'_E - (1 - \beta)x'_E - (1 + \beta)t'_E - (1 + \beta)x'_E \right)
\]
\[
-2x_E = \gamma \left( -2\beta t'_E - 2x'_E \right)
\]
Dividing by -2:
\[
x_E = \gamma \beta t'_E + \gamma x'_E
\]
Thus, we have verified the second equation:
\[
x_E = \gamma \beta t'_E + \gamma x'_E
\]


\subsubsection{Exercise}
We know that for a matrix $A$, and vectors $\mathbf{s}$ and $\mathbf{s}'$, the following is true
\[
\mathbf{s}' = A\mathbf{s} \then A^{-1}\mathbf{s}'=\mathbf{s}
\]
Now we look for the inverse of the Lorenz transform matrix. Since the transformation does not concern $y$ and $z$ axis, we treat it as a 2 by 2 matrix.
\[
A = \begin{bmatrix}
    \gamma & -\gamma\beta\\
    -\gamma\beta & \gamma
\end{bmatrix}
\]
The inverse of this matrix can be found as
\[
A^{-1} = \frac{1}{\abso{A}}\begin{bmatrix}
    \gamma & \gamma\beta\\
    \gamma\beta & \gamma
\end{bmatrix}
\]
Where $\abso{A}$ is the determinent of $A$, and is defined as
\[
\abso{A} = \gamma^2 -\gamma^2\beta^2 = \gamma^2\paren{1-\beta^2}
\]
But we know that $\frac{1}{\gamma^2} = 1 - \beta^2$, therefore
\[
\abso{A} = 1
\]
And the inverse of $A$ is simply 
\[
A^{-1} =\begin{bmatrix}
    \gamma & \gamma\beta\\
    \gamma\beta & \gamma
\end{bmatrix}
\]
Now we can add the unchanged $y$ and $z$ axis in to obtain the full inverse transformation.
\[
\mathbf{s} = \begin{bmatrix}
    \gamma & \gamma\beta & 0 & 0\\
    \gamma\beta & \gamma & 0 & 0\\
    0&0&1&0\\
    0&0&0&1
\end{bmatrix}\mathbf{s}'
\]

\subsection{BOX 2.4}
No Exercise in this box. Acknowledge that Lorenz transform is the same as a hyperbolic rotation, defined by a rotation matrix
\[
\cosh\theta = \frac{1}{\sqrt{1-\beta^2}} \equiv \gamma \quad \sinh\theta = \frac{\beta}{\sqrt{1-\beta^2}}=\gamma \beta
\]

\subsection{BOX 2.5}
\subsubsection{Exercise}
We would like to show that the \textit{spacetime interval} stays invariant throughout Lorenz transformation. First of all, we know that the spacetime interval is the magnitude of the 4-vector that we would denote as $\Delta s$. We know that the magnitude squared can be found by performing an inner product. We know that
\[
\mathbf{s'} = A\mathbf{s}
\]
Therefore
\[
(\mathbf{s}')^2 = \paren{\mathbf{s}'}^T\mathbf{s} = \paren{A\mathbf{s}}^TA\mathbf{s} = \mathbf{s}^TA^TA\mathbf{s}
\]
This would be true in 4-D Euclidean space, but we are working with a hyperbolic spacetime structure, so we have to incorporate another vector to adjust how we calculate the magnitude. The correct inner product should be
\[
\paren{\mathbf{s}'}^T\eta\mathbf{s}' = \paren{A\mathbf{s}}^T\eta A\mathbf{s} = \mathbf{s}^TA^T\eta A\mathbf{s}
\]
Where 
\[
\eta = \begin{bmatrix}
    -1&0&0&0\\
    0&1&0&0\\
    0&0&1&0\\
    0&0&0&1
\end{bmatrix}
\]
Now we can correctly obtain the magnitude of the 4-vector. Let's define
\[
\paren{\mathbf{s}'}^2 = \paren{\mathbf{s}'}^T\eta\mathbf{s}'
\]
We can examine the term $\mathbf{s}^TA^T$. Since $A$ is symmetric, then we have 
\[
\mathbf{s}^TA^T = \begin{bmatrix}
    t&x&y&z
\end{bmatrix}\begin{bmatrix}
\gamma & -\gamma\beta & 0 & 0 \\
-\gamma\beta & \gamma & 0 & 0 \\
0 & 0 & 1 & 0 \\
0 & 0 & 0 & 1
\end{bmatrix} = \left[\begin{matrix}
t\gamma-x\beta\gamma & x\gamma-t\beta\gamma & y & z
\end{matrix}\right]
\]
We see that this is simply
\[
\mathbf{s}^TA^T = \left[\begin{matrix}
t' & x' & y' & z'
\end{matrix}\right] = \paren{\mathbf{s}'}^T
\]
Also, by definition, we have that
\[
A\mathbf{s} = s'
\]
Therefore we have
\[
\paren{\mathbf{s}'}^T\eta\mathbf{s}' = \paren{\mathbf{s}'}^T\eta\mathbf{s}'
\]
This shows that the magnitude of spacetime 4-vector stays the same before and after the Lorenz transformation. Consequently, $\Delta s$ stays the same. 

\subsection{BOX 2.6}
    \subsubsection{Exercise}
    We start by writing out the Lorenz transformation between frame $S$ and $S'$.
    \[
    \Delta x' = \gamma\paren{\Delta x - \beta \Delta t}
    \]
    \[
    \Delta t' = \gamma\paren{\Delta t - \beta \Delta x}
    \]
    If the order of two events happened in opposite order, we have $\Delta t' < 0$. This means that
    \[
    \gamma\paren{\Delta t - \beta \Delta x} < 0
    \]
    Move things around, we have
    \[
    \frac{\Delta t}{\Delta x} < \beta 
    \]
    This is certainly possible if $\Delta x$ is sufficiently large, or $\beta$ is sufficiently large. And we know this is allowed since 
    \[
    -\Delta t^2 + \Delta x^2 > 0
    \]
    This tells us
    \[
    \Delta x^2 > \Delta t^2
    \]
    But if we examine timelike interval, we have
    \[
    \Delta x^2 < \Delta t^2
    \]
    This lead us back to where we were when we assumed $\Delta t' < 0$, or reversed order in time
    \[
    \frac{\Delta t}{\Delta x} > \beta 
    \]
    We know this is not possible since $\beta \in \sqbkt{0,1}$, this simply means nothing can travel faster than light, since $\beta$ is defined to be $\frac{v}{c}$. And we have shown that spacelike events does not have causal relationship, therefore the sequence of which they happened can be reversed, and vice versa. 

    \subsection{BOX 2.7}
        \subsubsection{Exercise}
        Since we are interested in the proper time on the clock carried on this differential step, we have
        \[
        dt' = \sqrt{dt^2 - dx^2 -dy^2 -dz^2}
        \]
        If we are to factor $dt^2$ out, we have
        \[
        dt' = dt\sqrt{1- \paren{\dydx{x^2}{t^2}+\dydx{y^2}{t^2}+\dydx{z^2}{t^2}}} = dt\sqrt{1-v^2}
        \]

    \subsection{BOX 2.8}
        \subsubsection{Exercise}
        We can use Lorenz transformation to prove length contraction by defining 2 events happening at the same time in frame $S$. This means a separation of time $0$. We have Lorenz transformation
        \[
        \Delta x' = \gamma\paren{\Delta x - \beta\Delta t}
        \]
        But we know that $\Delta t = 0$. Therefore
        \[
        \Delta x' = \gamma\Delta x
        \]
        This means
        \[
        \Delta x = \frac{1}{\gamma}\Delta x' = \Delta x'\sqrt{1-\beta^2}
        \]
        This shows that length is contracted compare to the proper length as measured in the frame where the events happened simultaneously. This means
        \[
        L = L_R\sqrt{1-\beta^2}
        \]

    \subsection{BOX 2.9}
        \subsubsection{Exercise}
        We can obtain $v_x$ by the following operation
        \[
        v_x' = \dydx{x'}{t'} = \frac{dx}{\gamma\paren{dt - \beta dx}} = \frac{dx}{dt}\frac{1}{\gamma\paren{1-\beta\dydx{x}{t}}} = \frac{v_x}{\gamma\paren{1-\beta v_x}} = \frac{v_x\paren{1-\beta^2}}{1-\beta v_x}
        \]








\end{document}