\documentclass[12pt]{article}
\pagestyle{empty}
\usepackage{amsmath, amssymb, amsthm}
\usepackage{latexsym, epsfig, ulem, cancel, multicol, hyperref}
\usepackage{graphicx, tikz, subfigure,pgfplots}
\usepackage[margin=1in]{geometry}
\setlength{\parindent}{0pt}
\usepackage{multirow}
\usepackage{mathtools}
\usepackage{verbatim}
\usepackage{tikz}
\usepackage{pgfplots}

\setlength{\parskip}{1ex}

\newcommand{\T}[0]{\top}
\newcommand{\F}[0]{\bot}
\newcommand{\liminfty}[1]{\lim_{#1 \to \infty}}
\newcommand{\limzero}[1]{\lim_{#1 \to 0}}
\newcommand{\limto}[1]{\lim_{#1}}
\newcommand{\Z}{\mathbb{Z}}
\newcommand{\R}{\mathbb{R}}
\newcommand{\C}{\mathbb{C}}
\newcommand{\Q}{\mathbb{Q}}
\newcommand{\odd}[0]{\mathbb{Z} - 2\mathbb{Z}}
\newcommand{\lineint}[1]{\int_{#1}}
\newcommand{\pypx}[2]{\frac{\partial #1}{\partial #2}}
\newcommand{\divg}{\nabla \cdot}
\newcommand{\curl}{\nabla \times}
\newcommand{\dydx}[2]{\frac{d #1}{d #2}}
\newcommand{\sqbkt}[1]{\left[ #1 \right]}
\newcommand{\paren}[1]{\left( #1 \right)}
\newcommand{\tribkt}[1]{\left< #1 \right>}
\newcommand{\abso}[1]{\left|#1 \right|}
\newcommand{\zero}{\{0\}}
\newcommand{\then}{\rightarrow}
\newcommand{\nonneg}{\Z^+ \cup \{0\}}
\DeclarePairedDelimiter\ceil{\lceil}{\rceil}
\DeclarePairedDelimiter\floor{\lfloor}{\rfloor}
\newcommand{\union}[2]{\bigcup_{#1}^{#2}}
\newcommand{\inter}[2]{\bigcap_{#1}^{#2}}
\newcommand{\openclose}[1]{\left( #1 \right]}
\newcommand{\closeopen}[1]{\left[ #1 \right)}
\newcommand{\compo}[2]{#1 e^{i #2}}
\newcommand{\laplase}{\bigtriangleup}
\newcommand{\bra}[1]{\left< #1 \right|}
\newcommand{\ket}[1]{\left| #1 \right>}
\newcommand{\braket}[2]{\left< #1 \mid #2 \right>}
\newcommand{\ketbra}[2]{\left| #1 \right> \left< #2 \right|}
\newcommand{\ketpsit}{\ket{\psi(t)}}
\newcommand{\ketphit}{\ket{\phi(t)}}
\newcommand{\ham}{\mathbf{H}}
\newcommand{\unx}{\hat{\mathbf{x}}}
\newcommand{\uny}{\hat{\mathbf{y}}}
\newcommand{\unz}{\hat{\mathbf{z}}}
\newcommand{\uni}{\hat{\mathbf{i}}}
\newcommand{\unj}{\hat{\mathbf{j}}}
\newcommand{\unk}{\hat{\mathbf{k}}}
\newcommand{\uns}{\hat{\mathbf{s}}}
\newcommand{\unr}{\hat{\mathbf{r}}}
\newcommand{\untheta}{\hat{\boldsymbol\theta}}
\newcommand{\unphi}{\hat{\boldsymbol\phi}}
\newcommand{\unrho}{\hat{\boldsymbol\rho}}



\newcommand{\wsnumber}{1}
\newcommand{\wstopic}{Vectors}
\pgfplotsset{
    every linear axis/.append style={
       axis x line=center,
       axis y line=center,
       xlabel={$x$},
       ylabel={$y$}
    },
    every axis plot/.append style={thick,mark=none}
}
\tikzset{
    point/.style={circle,draw,fill,minimum width=0.3ex,inner sep=0pt,outer sep=0pt},
    every label/.append style={black}
}


\usepackage[margin=1in]{geometry}
\usepackage{amsmath, amssymb, amsthm, graphicx, hyperref}
\usepackage{enumerate}
\usepackage{fancyhdr}
\usepackage{multirow, multicol}
\usepackage{tikz}
\pagestyle{fancy}
\fancyhead[RO]{Dennis Li}
\fancyhead[LO]{GR Independent Study}
\usepackage{comment}
\newif\ifshow
\showfalse

\ifshow
  \newenvironment{solution}{\textbf{Solution.}}{}
\else
  \excludecomment{solution}
\fi

\renewcommand{\thefootnote}{\fnsymbol{footnote}}
\usepackage{comment}


\newtheorem*{remark}{Remark}

\newcommand*{\GridSize}{4}

\newcommand*{\ColorCells}[1]{% #1 = list of x/y/color
  \foreach \x/\y/\color in {#1} {
    \node [fill=\color, draw=none, thick, minimum size=1cm] 
      at (\x-.5,\GridSize+0.5-\y) {};
    }%
}%

\begin{document}

\begin{center}
\ifshow
  \textbf{\Large GRW Workbook}\\
\else
  \textbf{\Large General Relativity Workbook}\\
\fi
Dennis Li\\Prof. Gabe\\
\end{center}

\hrule

\vspace{0.2cm}

\section{Introduction}
There are no boxes in this chapter, but selected problem will be done.

\section{Special Relativity}
\subsection{BOX 2.1}
\subsubsection{Exercise}
In the space below, prove the "only if" clause; that is, assume that S' is inertial and show that it must move at a constant velocity relative to S. Start by considering a free object at rest in S'.

\begin{proof}
    Suppose $S'$ is inertial and a \textbf{free} object is observed to at rest in $S'$. An object being free is mutually agreed upon by both $S$ and $S'$. By Newton's first law, the object will remain at rest in $S'$ since it is inertial by hypothesis. Now, if there is a second frame $S$ moving relative to $S'$, and observers in $S$ also observes the free object, then it will see the object moving along with $S'$ since the object is at rest with respect to $S'$. Since the object is free, it cannot participate in any interaction and therefore cannot change its state of motion. Therefore the only way for $S$ to be inertial is if it is moving at a constant velocity with respect to $S'$, otherwise it will observe a change in velocity in the free object.  
\end{proof}

\subsection{BOX 2.2}
\subsubsection{Exercise}
Verify that the following conversion is true. $M_\odot$ represents the solar mass.
\begin{align*}
g &= 1.09 \times 10^{-16} \, \text{m}^{-1} = 1 / (9.17 \times 10^{15} \, \text{m}) \approx 1 / (1 \, \text{ly})\\
G &= 7.426 \times 10^{-28} \, \text{m/kg} = 1477 \, \text{m} / M_\odot
\end{align*}
We start by verifying $g$. We start from $g = 9.8 \,\text{m/s$^2$}$. We introduce the conversion factor
\[
1\, \text{s} = 3\times 10^{8} \,\text{m}
\]
The constant can be rewritten as
\[
9.8 \times \frac{\text{m}}{\paren{3\times 10^{8}}^2\text{m}^2} =  1.09 \times 10^{-16} \, \text{m}^{-1} = \frac{1}{1.09 \times 10^{-16}\,\text{m}}
\]
We notice that
\[
1\,\text{ly} = \paren{365\times 24\times 60\times 60}\text{s} \times \paren{3\times 10^{8} \,\text{m/s}} \simeq 1\,\text{ly}
\]
Therefore we arrived 
\[
g = 1 / (1 \, \text{ly})
\]
Now we work on $G$. We know that 
\[
G = 6.67 \times 10^{-11} \frac{\text{N}\cdot\text{m}^2}{\text{kg}^2}
\]
We know that \textit{N} can be expressed in SI unit as $1\,$N$= 1\frac{\text{kg}\cdot\text{m}}{\text{s}^2}$. And the units simplifies to
\[
G = 6.67 \times 10^{-11} \frac{\text{m}^3}{\text{kg}\cdot \text{s}^2}
\]
We use the same conversion unit as before to rewrite the unit of time in meters, we will yield
\[
G = 6.67 \times 10^{-11} \frac{\text{m}}{\text{kg}\cdot \paren{3\times 10^{8}}^2}
\]
This gives us
\[
G = 7.426 \times 10^{-28}\, \text{m/kg}
\]
We know that the mass of the sun is
\[
M_\odot = 1.99 \times 10^{30}\,\text{kg}
\]
Then we can rewrite the expression as
\[
G = 1477 \, \text{m} / M_\odot
\]












\end{document}